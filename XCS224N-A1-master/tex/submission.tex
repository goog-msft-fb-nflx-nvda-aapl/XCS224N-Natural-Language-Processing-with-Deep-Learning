% This contents of this file will be inserted into the _Solutions version of the
% output tex document.  Here's an example:

% If assignment with subquestion (1.a) requires a written response, you will
% find the following flag within this document: <SCPD_SUBMISSION_TAG>_1a
% In this example, you would insert the LaTeX for your solution to (1.a) between
% the <SCPD_SUBMISSION_TAG>_1a flags.  If you also constrain your answer between the
% START_CODE_HERE and END_CODE_HERE flags, your LaTeX will be styled as a
% solution within the final document.

% Please do not use the '<SCPD_SUBMISSION_TAG>' character anywhere within your code.  As expected,
% that will confuse the regular expressions we use to identify your solution.
\def\assignmentnum{1 }
\def\assignmentname{Exploring Word Embeddings}
\def\assignmenttitle{XCS224N Assignment \assignmentnum \assignmentname}
\input{macros}
\begin{document}
\pagestyle{myheadings} \markboth{}{\assignmenttitle}

% <SCPD_SUBMISSION_TAG>_entire_submission

This handout includes space for every question that requires a written response.
Please feel free to use it to handwrite your solutions (legibly, please).  If
you choose to typeset your solutions, the |README.md| for this assignment includes
instructions to regenerate this handout with your typeset \LaTeX{} solutions.
\ruleskip

\LARGE
4.a
\normalsize

% <SCPD_SUBMISSION_TAG>_4a
\begin{answer}
  % ### START CODE HERE ###
  % ### END CODE HERE ###
\end{answer}
% <SCPD_SUBMISSION_TAG>_4a
\clearpage

\LARGE
4.b
\normalsize

% <SCPD_SUBMISSION_TAG>_4b
\begin{answer}
  % ### START CODE HERE ###
\begin{itemize}
        \item The columns of $\textbf{V}$ form an orthonormal basis for the row space of $\textbf{A}$, and thus $\textbf{V}^{T}\textbf{V} = \textbf{I}$. \\ 
        The $i$th column $\textbf{v}_{i}$ of $\textbf{V}$ is known as the $i$th right singular vector of $\textbf{A}$.  
\end{itemize}

Recall that for a linear subspace $W \subset \mathbb{R}^{p}$, the projection of some vector $\textbf{v}$ onto $W$ is defined as
\begin{equation}
\text{proj}_{W}(\textbf{v}) := \argmin_{w \in W} \vert \vert w - \textbf{v}\vert \vert _{2}^2
\label{proj_def}
\end{equation}

In the special case that $W$ is a one dimensional vector space spanned by some $\textbf{w}$, we call this projection the projection of  $\textbf{v}$ onto $\textbf{w}$ which has simple closed form. We give it below, where we have let $\hat{\textbf{w}} := \textbf{w}/\vert \vert \textbf{w}\vert \vert _2$ represent the unit vector pointing in the direction of $\textbf{w}$.

\begin{equation}
\textrm{proj}_{\textbf{w}}(\textbf{v}) = \bigg(\frac{\textbf{v}^{T}\textbf{w}}{\vert \vert \textbf{w}\vert \vert _2^2}\bigg) \textbf{w} = (\textbf{v}^{T}\hat{\textbf{w}}) \hat{\textbf{w}}
\label{proj_1d}
\end{equation}

  % ### END CODE HERE ###
\end{answer}
% <SCPD_SUBMISSION_TAG>_4b
\clearpage

\LARGE
4.c
\normalsize

% <SCPD_SUBMISSION_TAG>_4c
\begin{answer}
  % ### START CODE HERE ###
  % ### END CODE HERE ###
\end{answer}
% <SCPD_SUBMISSION_TAG>_4c
\clearpage

\LARGE
4.d
\normalsize

% <SCPD_SUBMISSION_TAG>_4d
\begin{answer}
  % ### START CODE HERE ###
  % ### END CODE HERE ###
\end{answer}
% <SCPD_SUBMISSION_TAG>_4d
\clearpage

% <SCPD_SUBMISSION_TAG>_entire_submission
\end{document}
